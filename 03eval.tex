\begin{enumerate}[i]
\item The solution of this problem should provided 0/1 labels for each pixel in every frame of the test videos. Label 0 denotes this pixel represents the background in the current frame while label 1 denotes the foreground. We could measure the accuracy of the provided solution by comparing every predicted label against the ground truth of the test videos. A successful solution should provide prediction results should be no worse than the incremental EM solution in \cite{friedman1997image}.
    Moreover, the solution provided by the teams must be online. In order to predict the result, we also need to evaluate the running time to produce labels for a single frame comparing against the total running time. The processing time for each frame for a successful online solution should be roughly the total time divided by the number of frames.
\item A incremental EM algorithm is mentioned in \cite{friedman1997image}.
\item There are many existing background substraction datasets. We could use a renderer engine to generate some small synthetic data as well as use a real-world dataset, i.e. the UCSD dataset\footnote{\url{http://www.svcl.ucsd.edu/projects/background_subtraction/ucsdbgsub_dataset.htm}}, for evaluating the performance.
\item The teams should load the video frames from disk and output all the labels for each video frame in a single text file. For example, when the video name is \texttt{A}, then the result for the 1st frame in \texttt{A} should be stored in \texttt{A1.txt}. The evaluation script should load all the labels from the text files and evaluate the predicted accuracy.
\item In the first stage, we will only deliver the small synthetic data. In the second stage, we deliver a small portion of the UCSD dataset to the teams and evaluate the performance using the remaining videos. Another choice for the second stage could be that for each video data in the UCSD dataset, we deliver only the first 25\% to the teams and using the whole video (or the remaining 75\%) for evaluation.
\end{enumerate} 