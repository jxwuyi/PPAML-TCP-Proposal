\newcommand{\secref}[1]{Section~\ref{#1}}
\newcommand{\figref}[1]{Figure~\ref{#1}}
\newcommand{\tabref}[1]{Table~\ref{#1}}
\newcommand{\eqref}[1]{Equation~(\ref{#1})}

\newcommand{\mbf}[1]{\mbox{{\bf #1}}}
\newcommand{\smbf}[1]{\mbox{{\scriptsize\bf #1}}}
\def\w{\mbf{w}}

%\section{Introduction}
%% {\centering\scriptsize\em
%%   Substantial portions of this text are reproduced from an
%%   author's paper~\cite{friedman:1997:uai}.}

%{\bf Movement Detection from Video:}
Recognizing and tracking moving
objects is one of the main uses of vision.
For computer vision, we can start with merely identifying the
foreground.
%first identify what part of each
%frame is due to moving objects.
%(subsequently, separating and tracking
%individuals across frames are also important subproblems)
That is, the task here is to decompose 
each frame of video into two parts: all of the moving objects, and
everything else.  Thankfully, in practice, approximate answers are
good enough.  \secref{sec:formal-problem} formally defines the problem,
but first, some background. 

For many years, the ``obvious'' approach has
been first to compute some stationary {\em background image}, and then to
identify the moving objects as those pixels in each frame that
differ significantly from the background. We will call this the
{\em background subtraction} approach. 
The details of the method are described briefly in
\secref{background-subtraction-section}.
To pick one application, the Roadwatch project at Berkeley,
background subtraction is overall an effective means of
locating and tracking moving vehicles in freeway
traffic~\cite{Koller+al:1994}.

However, in some important cases, background subtraction performs
poorly at vehicle detection: long shadows and heavy traffic. 
Shadows move, but treating them as real parts of the vehicles they
overlap leads to many problems, starting with reliably separating each
vehicle from one another.
Moreover, background subtraction tends to treat any
sufficiently slowly changing pixel as 
background, which is a problem in heavy traffic because vehicles
then too often disappear into the background.\footnote{Interestingly,
  many predators are also only able to see movement, so that
  it is possible to disappear just by freezing in place.}

These problems arise from the oversimplified view of the
task---detecting moving objects---that
background subtraction takes.  Namely, the approach assumes that
background never changes and foreground always changes.  However,
occasionally-variable background and occasionally-constant 
foreground are also possible.


Probabilistic techniques prove useful in developing a more
sophisticated approach.
Even just using a simple three-class (``background'', ``darkened'',
``occluded'') mixture model already significantly improves upon
background subtraction~\cite{friedman97uai}.  See
\secref{shadow-and-background-subtraction-section} for a summary of
that approach---which we take as baseline. 
Significant further improvement should be possible, perhaps
by applying more complicated probabilistic models.


%% To pick two
%% potential flaws of the reference work: it fails to consider nighttime
%% and deliberately ignores contiguity.
%% Much like shadows, the pixels brightened
%% by vehicles' headlights are easily misclassified as real parts of the
%% vehicles.  Using four classes
%% (``background'', ``brightened'', ``darkened'', ``occluded'') could help.
%% The reference work can be confused by a gray vehicle driving over a
%% gray manhole: it can mistakenly imagine that the manhole remains
%% visible when the vehicle drives over it.  Adding probabilistic contiguity
%% constraints would combat that and many related potential flaws.





\subsection{Notation}
\label{sec:notation}


\subsection{Foreground Video Segmentation}
\label{sec:formal-problem}

\subsection{Background Subtraction}
\label{background-subtraction-section}



\subsection{Probabilistic Shadow and Background Subtraction}
\label{shadow-and-background-subtraction-section}


